%%%%%%%%%%%%%%%%%%%%%%%%%%%%%%%%%%%%%%%%%
% Frequently Asked Questions
% LaTeX Template
% Version 1.0 (22/7/13)
%
% This template has been downloaded from:
% http://www.LaTeXTemplates.com
%
% Original author:
% Adam Glesser (adamglesser@gmail.com)
%
% License:
% CC BY-NC-SA 3.0 (http://creativecommons.org/licenses/by-nc-sa/3.0/)
%
%%%%%%%%%%%%%%%%%%%%%%%%%%%%%%%%%%%%%%%%%

\documentclass[11pt]{article}

\usepackage[margin=0.3in]{geometry} % Required to make the margins smaller to fit more content on each page
\usepackage[linkcolor=blue]{hyperref} % Required to create hyperlinks to questions from elsewhere in the document
\hypersetup{pdfborder={0 0 0}, colorlinks=true, urlcolor=blue} % Specify a color for hyperlinks
\usepackage{todonotes} % Required for the boxes that questions appear in
\usepackage{tocloft} % Required to give customize the table of contents to display questions
\usepackage{microtype} % Slightly tweak font spacing for aesthetics
\usepackage{palatino} % Use the Palatino font

\setlength\parindent{0pt} % Removes all indentation from paragraphs

% Create and define the list of questions
\newlistof{questions}{faq}{\large List of Frequently Asked Questions} % This creates a new table of contents-like environment that will output a file with extension .faq
\setlength\cftbeforefaqtitleskip{.5em} % Adjusts the vertical space between the title and subtitle
\setlength\cftafterfaqtitleskip{.5em} % Adjusts the vertical space between the subtitle and the first question
\setlength\cftparskip{.5em} % Adjusts the vertical space between questions in the list of questions

% Create the command used for questions
\newcommand{\question}[1] % This is what you will use to create a new question
{
\refstepcounter{questions} % Increases the questions counter, this can be referenced anywhere with \thequestions
\par\noindent % Creates a new unindented paragraph
\phantomsection % Needed for hyperref compatibility with the \addcontensline command
\addcontentsline{faq}{questions}{#1} % Adds the question to the list of questions
\todo[inline, color=white]{\textbf{#1}} % Uses the todonotes package to create a fancy box to put the question
\vspace{.5em} % White space after the question before the start of the answer
}

% Uncomment the line below to get rid of the trailing dots in the table of contents
%\renewcommand{\cftdot}{}

% Uncomment the two lines below to get rid of the numbers in the table of contents
%\let\Contentsline\contentsline
%\renewcommand\contentsline[3]{\Contentsline{#1}{#2}{}}

\begin{document}

%----------------------------------------------------------------------------------------
%	TITLE AND LIST OF QUESTIONS
%----------------------------------------------------------------------------------------

% \begin{center}
% \Huge{\bf \emph{A Template for FAQ's}} % Main title
% \end{center}

% \listofquestions % This prints the subtitle and a list of all of your questions

% \newpage % Comment this if you would like your questions and answers to start immediately after table of questions

%----------------------------------------------------------------------------------------
%	QUESTIONS AND ANSWERS
%----------------------------------------------------------------------------------------

\question{Marenostrum}\label{Marenostrum}

Copying files to and from marenostrum:
\begin{verbatim}
pscp <path/file name> bsc21774@mn1.bsc.es:<path/file name>
pscp bsc21774@mn1.bsc.es:<path/file name> <path/file name>
\end{verbatim}
For directories we add the option \texttt{-r} after  \texttt{pscp}.\\

Check out \texttt{Alya}:
\begin{verbatim}
mkdir <path/Alya>
svn co svn+ssh://bsc21774@mn1.bsc.es/gpfs/projects/bsc21/svnroot/Alya/Trunk <path/Alya>
\end{verbatim}


Search, load and remove modules:
\begin{verbatim}
module list
module avail
module load <module name>
module purge <module name>
\end{verbatim}

Avoid ssh's verification for known hosts, in:
\begin{verbatim}
/etc/ssh/ssh_config
\end{verbatim}

\begin{verbatim}
Host mn1.bsc.es
   StrictHostKeyChecking no
   UserKnownHostsFile=/dev/null
\end{verbatim}

Avoid ssh's password:
\begin{verbatim}
    ssh-keygen -b 4096 -t rsa
    ssh-copy-id bsc21774@mn1.bsc.es
\end{verbatim}

Create a repository on Marenostrum for running thing there updating code only doing \verb push and \verb pull .

On Marenostrum:
\begin{verbatim}
 mkdir sputnik.git
 cd sputnik.git
 git init --bare
\end{verbatim}

On Local computer:
\begin{verbatim}
 git remote add mn1 bsc21774@mn1.bsc.es:/gpfs/home/bsc21/bsc21774/codes/sputnik.git/
 git push --all mn1
\end{verbatim}

On Marenostrum:
\begin{verbatim}
 git clone sputnik.git
\end{verbatim}

%------------------------------------------------

\question{GDB}\label{GDB}

On \texttt{\$HOME} create a \texttt{.gdbinit} file and write:

\begin{verbatim}
set history save on
\end{verbatim}


Breakpoint at line 13 of main.c:
\begin{verbatim}
b main.c : 13 
\end{verbatim}

Breakpoint at function (enters inside:
\begin{verbatim}
b function
\end{verbatim}

Breakpoint condicional:
\begin{verbatim}
b main.c : 13 if a==2
\end{verbatim}

Info Brakpoints:
\begin{verbatim}
i b  
\end{verbatim}

Delete Breakpoint \#1:
\begin{verbatim}
d 1
\end{verbatim}

Run:
\begin{verbatim}
r
\end{verbatim}

Continue to the next breakpoint:
\begin{verbatim}
c
\end{verbatim}

Finish (scapes from the current function):
\begin{verbatim}
f
\end{verbatim}

Print variable ``a'':
\begin{verbatim}
p a
\end{verbatim}


%------------------------------------------------

\question{Disks}\label{Disks}

List connected disks (as root):
\begin{verbatim}
fdisk -l 
umount /dev/<device>
mkdir  /media/<folder to mount>
mount -o rw,uid=1000,gid=1000,user,exec,umask=003 /dev/<device> /media/<folder to mount>
\end{verbatim}

%------------------------------------------------

\question{Screen}\label{Screen}

For dual screens (BSC = laptop + monitor):
\begin{verbatim}
xrandr
xrandr --output LVDS1 --mode 1280x800
xrandr --output DP1   --mode 1920x1080 --left-of LVDS1
\end{verbatim}

To put the correct devices and sizes \texttt{xrandr} alone list the devices available.

%------------------------------------------------

\question{PDF}\label{PDF}

Diminish \texttt{pdf} quality:

\begin{verbatim}
//convert -density 200x200 -quality 60 -compress jpeg input.pdf output.pdf
\end{verbatim}

Convert \texttt{jpg} to \texttt{pdf}:

\begin{verbatim}
convert input.jpg output.pdf
\end{verbatim}

Extract pages:
\begin{verbatim}
pdftk input.pdf cat 12-15 output output.pdf
\end{verbatim}

Rotate pages:
\begin{verbatim}
pdftk padron1.pdf cat 1-endwest output padron2.pdf
pdftk A=iscrizione_1.pdf cat A1-4 A5east output iscrizione_2.pdf
\end{verbatim}

Join:
\begin{verbatim}
pdfjoin a.pdf b.pdf c.pdf -o output.pdf
\end{verbatim}
%------------------------------------------------
\question{GDB}\label{GDB}

\begin{verbatim}
gdb --args program <args>
\end{verbatim}
Set breakpoints:
\begin{verbatim}
b <file_name>:#line
b <function_name>
run      (alias:r)
continue (alias:c)
p <variable_name>
\end{verbatim}
Continue until the current loop finish:
\begin{verbatim}
until    
\end{verbatim}


%------------------------------------------------

\question{GIT}\label{PDF}

\begin{verbatim}
git status
git diff <file name>
\end{verbatim}

%------------------------------------------------

\question{Alya}\label{PDF}

gmsh2alya
\begin{verbatim}
gmsh2alya <mesh_file> --bsc=boundaries
\end{verbatim}

%------------------------------------------------

\question{Mercurial}\label{Mercurial}

Basic commands
\begin{verbatim}
hg init
hg add <filename>
hg commit -m "message"
hg push
hg pull
\end{verbatim}

File \verb .hgignore :
\begin{verbatim}
syntax: glob

obj/*.o
*.vtk
*.msh
*.png
\end{verbatim}  



%------------------------------------------------

\question{Xmodmap}\label{Xmodmap}

\begin{verbatim}
xmodmap -e "remove Lock= Caps_Lock" 
xmodmap -e "keysym Caps_Lock = Escape" 
\end{verbatim}
%------------------------------------------------

\question{xterm}\label{xterm}

\begin{verbatim}
su
xrdb -merge ~/.Xresources
\end{verbatim}

%------------------------------------------------
\question{Pyxplot}\label{Pyxplot}

\begin{verbatim}
set terminal landscape pdf enlarge
set output 'energy.pdf'

set xlabel "Displacement [m]"
set ylabel "Energy"
set key bottom right

plot 'energy.dat' u 1:($4)    w l every 1::1::1000 title "External",\
     'energy.dat' u 1:($5)    w P every 50::1::1000 title "Internal",\
     'energy.dat' u 1:($2+$3) w p every 57::1::1000 title "E+P"
     
set output ’uniaxial_dipl.pdf’
set xlabel "$u$"
set ylabel "$v$"
set key top right
set grid   
plot 'file123.dat' u 1:2 w l title "Solidz (TL)",\
     'file123.dat' u 1:3 w p pt 1 ps 0.7  title "Code\_Aster (TL)"  ,\
     'ostero_uniaxial/disp.dat' u 1:2 w lp pt 1 ps 0.7 title "Ostero (SD)"
\end{verbatim}

%------------------------------------------------
\question{Gnuplot}\label{Gnuplot}

Plot a matrix:
\begin{verbatim}
plot 'matrix.dat' matrix with image
\end{verbatim}

%------------------------------------------------

\question{Kate}\label{Kate}

Highlighting folder KDE 4:

\begin{verbatim}
~/.kde/share/apps/katepart/syntax/
\end{verbatim}

Highlighting folder KDE 5:

\begin{verbatim}
~/.local/share/katepart5/syntax/
\end{verbatim}

%------------------------------------------------
\newpage
\question{Python}\label{Python}

Program header:

\begin{verbatim}
#!/usr/bin/env python
\end{verbatim}

Simple for loop:
\begin{verbatim}
for i in range(10):
   print i
\end{verbatim}

String operations:
\begin{verbatim}
name = name.replace('"','')
\end{verbatim}

Numpy utilities:
\begin{verbatim}
import numpy as np

vector = [1,2]
np.linalg.norm(vector)
\end{verbatim}

%------------------------------------------------
\newpage
\question{Awk}\label{Awk}

Invert rows:

\begin{verbatim}
#!/bin/bash
awk '
BEGIN {i=1}      
{ for (j=1; j<=NF; j++)  
       a[i,j] = $j
  i++;}
NF>p { p = NF }
END {
    for(i=NR; i>0; i--) {
        str=""; 
        for(j=0; j<=p; j++)
            str=str" "a[i,j];      
        str=str;
        print str
    }
}' $1 > $2 
\end{verbatim}

%----------------------------------------------------------------------------------------

\end{document}
